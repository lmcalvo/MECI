\documentclass[]{article}
\usepackage{lmodern}
\usepackage{amssymb,amsmath}
\usepackage{ifxetex,ifluatex}
\usepackage{fixltx2e} % provides \textsubscript
\ifnum 0\ifxetex 1\fi\ifluatex 1\fi=0 % if pdftex
  \usepackage[T1]{fontenc}
  \usepackage[utf8]{inputenc}
\else % if luatex or xelatex
  \ifxetex
    \usepackage{mathspec}
  \else
    \usepackage{fontspec}
  \fi
  \defaultfontfeatures{Ligatures=TeX,Scale=MatchLowercase}
\fi
% use upquote if available, for straight quotes in verbatim environments
\IfFileExists{upquote.sty}{\usepackage{upquote}}{}
% use microtype if available
\IfFileExists{microtype.sty}{%
\usepackage{microtype}
\UseMicrotypeSet[protrusion]{basicmath} % disable protrusion for tt fonts
}{}
\usepackage[margin=1in]{geometry}
\usepackage{hyperref}
\hypersetup{unicode=true,
            pdftitle={Algoritmo de Gibbs},
            pdfauthor={Luis Miguel Calvo Magaz and Adrián Poncela Gómez},
            pdfborder={0 0 0},
            breaklinks=true}
\urlstyle{same}  % don't use monospace font for urls
\usepackage{color}
\usepackage{fancyvrb}
\newcommand{\VerbBar}{|}
\newcommand{\VERB}{\Verb[commandchars=\\\{\}]}
\DefineVerbatimEnvironment{Highlighting}{Verbatim}{commandchars=\\\{\}}
% Add ',fontsize=\small' for more characters per line
\usepackage{framed}
\definecolor{shadecolor}{RGB}{248,248,248}
\newenvironment{Shaded}{\begin{snugshade}}{\end{snugshade}}
\newcommand{\KeywordTok}[1]{\textcolor[rgb]{0.13,0.29,0.53}{\textbf{#1}}}
\newcommand{\DataTypeTok}[1]{\textcolor[rgb]{0.13,0.29,0.53}{#1}}
\newcommand{\DecValTok}[1]{\textcolor[rgb]{0.00,0.00,0.81}{#1}}
\newcommand{\BaseNTok}[1]{\textcolor[rgb]{0.00,0.00,0.81}{#1}}
\newcommand{\FloatTok}[1]{\textcolor[rgb]{0.00,0.00,0.81}{#1}}
\newcommand{\ConstantTok}[1]{\textcolor[rgb]{0.00,0.00,0.00}{#1}}
\newcommand{\CharTok}[1]{\textcolor[rgb]{0.31,0.60,0.02}{#1}}
\newcommand{\SpecialCharTok}[1]{\textcolor[rgb]{0.00,0.00,0.00}{#1}}
\newcommand{\StringTok}[1]{\textcolor[rgb]{0.31,0.60,0.02}{#1}}
\newcommand{\VerbatimStringTok}[1]{\textcolor[rgb]{0.31,0.60,0.02}{#1}}
\newcommand{\SpecialStringTok}[1]{\textcolor[rgb]{0.31,0.60,0.02}{#1}}
\newcommand{\ImportTok}[1]{#1}
\newcommand{\CommentTok}[1]{\textcolor[rgb]{0.56,0.35,0.01}{\textit{#1}}}
\newcommand{\DocumentationTok}[1]{\textcolor[rgb]{0.56,0.35,0.01}{\textbf{\textit{#1}}}}
\newcommand{\AnnotationTok}[1]{\textcolor[rgb]{0.56,0.35,0.01}{\textbf{\textit{#1}}}}
\newcommand{\CommentVarTok}[1]{\textcolor[rgb]{0.56,0.35,0.01}{\textbf{\textit{#1}}}}
\newcommand{\OtherTok}[1]{\textcolor[rgb]{0.56,0.35,0.01}{#1}}
\newcommand{\FunctionTok}[1]{\textcolor[rgb]{0.00,0.00,0.00}{#1}}
\newcommand{\VariableTok}[1]{\textcolor[rgb]{0.00,0.00,0.00}{#1}}
\newcommand{\ControlFlowTok}[1]{\textcolor[rgb]{0.13,0.29,0.53}{\textbf{#1}}}
\newcommand{\OperatorTok}[1]{\textcolor[rgb]{0.81,0.36,0.00}{\textbf{#1}}}
\newcommand{\BuiltInTok}[1]{#1}
\newcommand{\ExtensionTok}[1]{#1}
\newcommand{\PreprocessorTok}[1]{\textcolor[rgb]{0.56,0.35,0.01}{\textit{#1}}}
\newcommand{\AttributeTok}[1]{\textcolor[rgb]{0.77,0.63,0.00}{#1}}
\newcommand{\RegionMarkerTok}[1]{#1}
\newcommand{\InformationTok}[1]{\textcolor[rgb]{0.56,0.35,0.01}{\textbf{\textit{#1}}}}
\newcommand{\WarningTok}[1]{\textcolor[rgb]{0.56,0.35,0.01}{\textbf{\textit{#1}}}}
\newcommand{\AlertTok}[1]{\textcolor[rgb]{0.94,0.16,0.16}{#1}}
\newcommand{\ErrorTok}[1]{\textcolor[rgb]{0.64,0.00,0.00}{\textbf{#1}}}
\newcommand{\NormalTok}[1]{#1}
\usepackage{graphicx,grffile}
\makeatletter
\def\maxwidth{\ifdim\Gin@nat@width>\linewidth\linewidth\else\Gin@nat@width\fi}
\def\maxheight{\ifdim\Gin@nat@height>\textheight\textheight\else\Gin@nat@height\fi}
\makeatother
% Scale images if necessary, so that they will not overflow the page
% margins by default, and it is still possible to overwrite the defaults
% using explicit options in \includegraphics[width, height, ...]{}
\setkeys{Gin}{width=\maxwidth,height=\maxheight,keepaspectratio}
\IfFileExists{parskip.sty}{%
\usepackage{parskip}
}{% else
\setlength{\parindent}{0pt}
\setlength{\parskip}{6pt plus 2pt minus 1pt}
}
\setlength{\emergencystretch}{3em}  % prevent overfull lines
\providecommand{\tightlist}{%
  \setlength{\itemsep}{0pt}\setlength{\parskip}{0pt}}
\setcounter{secnumdepth}{0}
% Redefines (sub)paragraphs to behave more like sections
\ifx\paragraph\undefined\else
\let\oldparagraph\paragraph
\renewcommand{\paragraph}[1]{\oldparagraph{#1}\mbox{}}
\fi
\ifx\subparagraph\undefined\else
\let\oldsubparagraph\subparagraph
\renewcommand{\subparagraph}[1]{\oldsubparagraph{#1}\mbox{}}
\fi

%%% Use protect on footnotes to avoid problems with footnotes in titles
\let\rmarkdownfootnote\footnote%
\def\footnote{\protect\rmarkdownfootnote}

%%% Change title format to be more compact
\usepackage{titling}

% Create subtitle command for use in maketitle
\newcommand{\subtitle}[1]{
  \posttitle{
    \begin{center}\large#1\end{center}
    }
}

\setlength{\droptitle}{-2em}

  \title{Algoritmo de Gibbs}
    \pretitle{\vspace{\droptitle}\centering\huge}
  \posttitle{\par}
    \author{Luis Miguel Calvo Magaz and Adrián Poncela Gómez}
    \preauthor{\centering\large\emph}
  \postauthor{\par}
      \predate{\centering\large\emph}
  \postdate{\par}
    \date{7 de marzo de 2019}


\begin{document}
\maketitle

\subsection{Introducción}\label{introduccion}

Considera un modelo de regresión con efectos aleatorios para analizar
los datos del ejemplo D (ratas.xls: contiene las observaciones del peso
en 5 momentos del tiempo : 8,15,22,29 y 36 días). Considera que tanto el
término independiente como el coeficiente de la variable explicativa
(tiempo centrado \((x-\bar x)\)) son a su vez variables aleatorias
normales que dependen de hiperparámetros.

\subsection{Modelo en términos de los parámetros e
hiperparámetros.}\label{modelo-en-terminos-de-los-parametros-e-hiperparametros.}

Modelo jerárquico normal:
\[y_{ij}\;=\; Peso\;de\;la\;rata\;i\;en\;el\;día\;x_j\;,\;i\in\{1,2,...,30\}\;,\;x_j\in\{8,15,22,29,36\}\]
\[x_j\;=\;Momento\;de\;tiempo\;j\;,\;j\in\{1,2,3,4,5\}\;,\;x_j\in\{8,15,22,29,36\}\]
\[y_{ij} \sim Normal(\alpha_i + \beta_i·(x_j-\bar{x}) ,\;\sigma_y^2)\]\\
donde:

\[\alpha_i \sim Normal(\mu_\alpha,\sigma_\alpha^2)\]
\[\beta_i \sim Normal(\mu_\beta,\sigma_\beta^2)\]

\subsection{Resumen de los datos}\label{resumen-de-los-datos}

\begin{verbatim}
##        tiempo1 tiempo2 tiempo3 tiempo4 tiempo5
## rata1      151     199     246     283     320
## rata2      145     199     249     293     354
## rata3      147     214     263     312     328
## rata4      155     200     237     272     297
## rata5      135     188     230     280     323
## rata6      159     210     252     298     331
## rata7      141     189     231     275     305
## rata8      159     201     248     297     338
## rata9      177     236     285     350     376
## rata10     134     182     220     260     296
## rata11     160     208     261     313     352
## rata12     143     188     220     273     314
## rata13     154     200     244     289     325
## rata14     171     221     270     326     358
## rata15     163     216     242     281     312
## rata16     160     207     248     288     324
## rata17     142     187     234     280     316
## rata18     156     203     243     283     317
## rata19     157     212     259     307     336
## rata20     152     203     246     286     321
## rata21     154     205     253     298     334
## rata22     139     190     225     267     302
## rata23     146     191     229     272     302
## rata24     157     211     250     285     323
## rata25     132     185     237     286     331
## rata26     160     207     257     303     345
## rata27     169     216     261     295     333
## rata28     157     205     248     289     316
## rata29     137     180     219     258     291
## rata30     153     200     244     286     324
\end{verbatim}

Medias:

\begin{verbatim}
##  tiempo1  tiempo2  tiempo3  tiempo4  tiempo5 
## 152.1667 201.7667 245.0333 289.5000 324.8000
\end{verbatim}

Varianzas:

\begin{verbatim}
##  tiempo1  tiempo2  tiempo3  tiempo4  tiempo5 
## 124.6264 160.5989 236.2402 367.0172 378.6483
\end{verbatim}

\subsection{Implementación del Modelo jerárquico en
R:}\label{implementacion-del-modelo-jerarquico-en-r}

\subsubsection{Funciones de las Distribuciones condicionadas de los
parámetros}\label{funciones-de-las-distribuciones-condicionadas-de-los-parametros}

\begin{figure}
\centering
\includegraphics{./alpha_condicionada.png}
\caption{}
\end{figure}

\begin{Shaded}
\begin{Highlighting}[]
\NormalTok{alpha_condicionada <-}\StringTok{ }\ControlFlowTok{function}\NormalTok{(mu_alpha,sigma2_alpha,peso,beta,tiempos)\{}
\NormalTok{  J <-}\StringTok{ }\KeywordTok{length}\NormalTok{(peso)}
\NormalTok{  media <-}\StringTok{ }\DecValTok{0}
\NormalTok{  varianza <-}\StringTok{ }\DecValTok{0}
\NormalTok{  sumatorio <-}\StringTok{ }\DecValTok{0}
  \ControlFlowTok{for}\NormalTok{(j }\ControlFlowTok{in} \DecValTok{1}\OperatorTok{:}\NormalTok{J)\{}
\NormalTok{      sumatorio <-}\StringTok{ }\NormalTok{sumatorio }\OperatorTok{+}\StringTok{ }\NormalTok{(peso[j]}\OperatorTok{-}\NormalTok{beta}\OperatorTok{*}\NormalTok{tiempos[j])}
\NormalTok{  \}}
\NormalTok{  media <-}\StringTok{ }\NormalTok{(mu_alpha}\OperatorTok{/}\NormalTok{sigma2_alpha) }\OperatorTok{+}\StringTok{ }\NormalTok{(sumatorio}\OperatorTok{/}\NormalTok{((}\DecValTok{1}\OperatorTok{/}\NormalTok{sigma2_alpha)}\OperatorTok{+}\NormalTok{(J}\OperatorTok{/}\NormalTok{sigma2_y)))}
\NormalTok{  varianza <-}\StringTok{ }\DecValTok{1}\OperatorTok{/}\NormalTok{((}\DecValTok{1}\OperatorTok{/}\NormalTok{sigma2_alpha)}\OperatorTok{+}\NormalTok{(J}\OperatorTok{/}\NormalTok{sigma2_y))}
  \KeywordTok{return}\NormalTok{(}\KeywordTok{rnorm}\NormalTok{(}\DecValTok{1}\NormalTok{,media,}\KeywordTok{sqrt}\NormalTok{(varianza)))}
\NormalTok{\}}
\end{Highlighting}
\end{Shaded}

\begin{figure}
\centering
\includegraphics{./beta_condicionada.png}
\caption{}
\end{figure}

\begin{Shaded}
\begin{Highlighting}[]
\NormalTok{beta_condicionada <-}\StringTok{ }\ControlFlowTok{function}\NormalTok{(mu_beta,sigma2_beta,peso,alpha,tiempos)\{}
\NormalTok{  J <-}\StringTok{ }\KeywordTok{length}\NormalTok{(peso)}
\NormalTok{  media <-}\StringTok{ }\DecValTok{0}
\NormalTok{  varianza <-}\StringTok{ }\DecValTok{0}
\NormalTok{  sumatorio <-}\StringTok{ }\DecValTok{0}
  \ControlFlowTok{for}\NormalTok{(j }\ControlFlowTok{in} \DecValTok{1}\OperatorTok{:}\NormalTok{J)\{}
\NormalTok{    sumatorio <-}\StringTok{ }\NormalTok{sumatorio}\OperatorTok{+}\NormalTok{(peso[j]}\OperatorTok{*}\NormalTok{tiempos[j]}\OperatorTok{-}\NormalTok{alpha}\OperatorTok{*}\NormalTok{tiempos[j])}
\NormalTok{  \}}
\NormalTok{  media <-}\StringTok{ }\NormalTok{((mu_beta}\OperatorTok{/}\NormalTok{sigma2_beta) }\OperatorTok{+}\StringTok{ }\NormalTok{(sumatorio}\OperatorTok{/}\NormalTok{sigma2_y))}\OperatorTok{/}\NormalTok{((}\DecValTok{1}\OperatorTok{/}\NormalTok{sigma2_beta)}\OperatorTok{+}\NormalTok{(}\KeywordTok{sum}\NormalTok{(tiempos}\OperatorTok{^}\DecValTok{2}\NormalTok{)}\OperatorTok{/}\NormalTok{sigma2_y))}
\NormalTok{  varianza <-}\StringTok{ }\DecValTok{1}\OperatorTok{/}\NormalTok{((}\DecValTok{1}\OperatorTok{/}\NormalTok{sigma2_beta)}\OperatorTok{+}\NormalTok{(}\KeywordTok{sum}\NormalTok{(tiempos}\OperatorTok{^}\DecValTok{2}\NormalTok{)}\OperatorTok{/}\NormalTok{sigma2_y))}
  \KeywordTok{return}\NormalTok{(}\KeywordTok{rnorm}\NormalTok{(}\DecValTok{1}\NormalTok{,media,}\KeywordTok{sqrt}\NormalTok{(varianza)))}
\NormalTok{\}}
\end{Highlighting}
\end{Shaded}

\begin{figure}
\centering
\includegraphics{./sigma_y_condicionada.png}
\caption{}
\end{figure}

\begin{Shaded}
\begin{Highlighting}[]
\NormalTok{sigma2_y_condicionada <-}\StringTok{ }\ControlFlowTok{function}\NormalTok{(Y,alpha,beta,X)\{}
\NormalTok{  n <-}\StringTok{ }\KeywordTok{dim}\NormalTok{(Y)[}\DecValTok{1}\NormalTok{]}
\NormalTok{  J <-}\StringTok{ }\KeywordTok{dim}\NormalTok{(Y)[}\DecValTok{2}\NormalTok{]}
\NormalTok{  media <-}\StringTok{ }\NormalTok{n}\OperatorTok{*}\NormalTok{J}\OperatorTok{/}\DecValTok{2}
\NormalTok{  sumatorio <-}\StringTok{ }\FloatTok{0.0}
  \ControlFlowTok{for}\NormalTok{(j }\ControlFlowTok{in} \DecValTok{1}\OperatorTok{:}\NormalTok{J)\{}
    \ControlFlowTok{for}\NormalTok{(i }\ControlFlowTok{in} \DecValTok{1}\OperatorTok{:}\NormalTok{n)\{}
\NormalTok{      sumatorio <-}\StringTok{ }\NormalTok{sumatorio }\OperatorTok{+}\StringTok{ }\NormalTok{(Y[i,j] }\OperatorTok{-}\StringTok{ }\NormalTok{(alpha[i] }\OperatorTok{+}\StringTok{ }\NormalTok{beta[i]}\OperatorTok{*}\NormalTok{X[j]))}\OperatorTok{^}\DecValTok{2}
\NormalTok{    \}}
\NormalTok{  \}}
\NormalTok{  dispersion <-}\StringTok{ }\NormalTok{sumatorio}\OperatorTok{/}\DecValTok{2}
\NormalTok{  gamma_inversa <-}\StringTok{ }\KeywordTok{rigamma}\NormalTok{(}\DataTypeTok{n=}\DecValTok{1}\NormalTok{,}\DataTypeTok{alpha =}\NormalTok{ media,}\DataTypeTok{beta =}\NormalTok{ dispersion)}
  \KeywordTok{return}\NormalTok{(gamma_inversa)}
\NormalTok{\}}
\end{Highlighting}
\end{Shaded}

\begin{figure}
\centering
\includegraphics{./sigma_a_condicionada.png}
\caption{}
\end{figure}

\begin{Shaded}
\begin{Highlighting}[]
\NormalTok{sigma2_alpha_condicionada <-}\StringTok{ }\ControlFlowTok{function}\NormalTok{(Y,alpha,mu_alpha)\{}
\NormalTok{  n <-}\StringTok{ }\KeywordTok{dim}\NormalTok{(Y)[}\DecValTok{1}\NormalTok{]}
\NormalTok{  media <-}\StringTok{ }\NormalTok{n}\OperatorTok{/}\DecValTok{2}
\NormalTok{  sumatorio <-}\StringTok{ }\DecValTok{0}
  \ControlFlowTok{for}\NormalTok{(i }\ControlFlowTok{in} \DecValTok{1}\OperatorTok{:}\NormalTok{n)\{}
\NormalTok{    sumatorio <-}\StringTok{ }\NormalTok{sumatorio }\OperatorTok{+}\StringTok{ }\NormalTok{(alpha[i] }\OperatorTok{-}\StringTok{ }\NormalTok{mu_alpha)}\OperatorTok{^}\DecValTok{2}
\NormalTok{  \}}
\NormalTok{  dispersion <-}\StringTok{ }\NormalTok{sumatorio}\OperatorTok{/}\DecValTok{2}
\NormalTok{  gamma_inversa <-}\StringTok{ }\KeywordTok{rigamma}\NormalTok{(}\DataTypeTok{n=}\DecValTok{1}\NormalTok{,}\DataTypeTok{alpha=}\NormalTok{media,}\DataTypeTok{beta =}\NormalTok{dispersion)}
  \KeywordTok{return}\NormalTok{(gamma_inversa)}
\NormalTok{\}}
\end{Highlighting}
\end{Shaded}

\begin{figure}
\centering
\includegraphics{./sigma_b_condicionada.png}
\caption{}
\end{figure}

\begin{Shaded}
\begin{Highlighting}[]
\NormalTok{sigma2_beta_condicionada <-}\StringTok{ }\ControlFlowTok{function}\NormalTok{(Y,beta,mu_beta)\{}
\NormalTok{  n <-}\StringTok{ }\KeywordTok{dim}\NormalTok{(Y)[}\DecValTok{1}\NormalTok{]}
\NormalTok{  media <-}\StringTok{ }\NormalTok{n}\OperatorTok{/}\DecValTok{2}
\NormalTok{  sumatorio <-}\StringTok{ }\DecValTok{0}
  \ControlFlowTok{for}\NormalTok{(i }\ControlFlowTok{in} \DecValTok{1}\OperatorTok{:}\NormalTok{n)\{}
\NormalTok{    sumatorio <-}\StringTok{ }\NormalTok{sumatorio }\OperatorTok{+}\StringTok{ }\NormalTok{(beta[i] }\OperatorTok{-}\StringTok{ }\NormalTok{mu_beta)}\OperatorTok{^}\DecValTok{2}
\NormalTok{  \}}
\NormalTok{  dispersion <-}\StringTok{ }\NormalTok{sumatorio}\OperatorTok{/}\DecValTok{2}
\NormalTok{  gamma_inversa <-}\StringTok{ }\KeywordTok{rigamma}\NormalTok{(}\DataTypeTok{n=}\DecValTok{1}\NormalTok{,}\DataTypeTok{alpha=}\NormalTok{media,}\DataTypeTok{beta =}\NormalTok{dispersion)}
  \KeywordTok{return}\NormalTok{(gamma_inversa)}
\NormalTok{\}}
\end{Highlighting}
\end{Shaded}

\begin{figure}
\centering
\includegraphics{./mu_a_condicionada.png}
\caption{}
\end{figure}

\begin{Shaded}
\begin{Highlighting}[]
\NormalTok{mu_alpha_condicionada <-}\StringTok{ }\ControlFlowTok{function}\NormalTok{(Y,alpha,sigma2_alpha)\{}
\NormalTok{  n <-}\StringTok{ }\KeywordTok{dim}\NormalTok{(Y)[}\DecValTok{1}\NormalTok{]}
\NormalTok{  media <-}\StringTok{ }\KeywordTok{sum}\NormalTok{(alpha)}\OperatorTok{/}\NormalTok{n}
\NormalTok{  varianza <-}\StringTok{ }\NormalTok{sigma2_alpha}\OperatorTok{/}\NormalTok{n}
  \KeywordTok{return}\NormalTok{(}\KeywordTok{rnorm}\NormalTok{(}\DecValTok{1}\NormalTok{,media,}\KeywordTok{sqrt}\NormalTok{(varianza)))}
\NormalTok{\}}
\end{Highlighting}
\end{Shaded}

\begin{figure}
\centering
\includegraphics{./mu_a_condicionada.png}
\caption{}
\end{figure}

\begin{Shaded}
\begin{Highlighting}[]
\NormalTok{mu_beta_condicionada <-}\StringTok{ }\ControlFlowTok{function}\NormalTok{(Y,beta,sigma2_beta)\{}
\NormalTok{  n <-}\StringTok{ }\KeywordTok{dim}\NormalTok{(Y)[}\DecValTok{1}\NormalTok{]}
\NormalTok{  media <-}\StringTok{ }\KeywordTok{sum}\NormalTok{(beta)}\OperatorTok{/}\NormalTok{n}
\NormalTok{  varianza <-}\StringTok{ }\NormalTok{sigma2_beta}\OperatorTok{/}\NormalTok{n}
  \KeywordTok{return}\NormalTok{(}\KeywordTok{rnorm}\NormalTok{(}\DecValTok{1}\NormalTok{,media,}\KeywordTok{sqrt}\NormalTok{(varianza)))}
\NormalTok{\}}
\end{Highlighting}
\end{Shaded}

\subsubsection{Implementación del algoritmo de Gibbs en
R:}\label{implementacion-del-algoritmo-de-gibbs-en-r}

Buscamos el mayor nivel de actualización, por lo que actualizamos todos
los parámetros en cada iteracción para cada una de las ratas.

\begin{Shaded}
\begin{Highlighting}[]
\NormalTok{gibbs <-}\StringTok{ }\ControlFlowTok{function}\NormalTok{(datos,tiempos,numero_iteraciones,alpha,beta,sigma2_y,sigma2_alpha,sigma2_beta,mu_alpha,mu_beta)\{}
\NormalTok{  numero_parametros<-}\KeywordTok{length}\NormalTok{(alpha)}\OperatorTok{+}\KeywordTok{length}\NormalTok{(beta)}\OperatorTok{+}\DecValTok{5}
\NormalTok{  parametros<-}\KeywordTok{matrix}\NormalTok{(}\KeywordTok{rep}\NormalTok{(}\FloatTok{0.00}\NormalTok{,numero_parametros}\OperatorTok{*}\NormalTok{numero_iteraciones),}\KeywordTok{c}\NormalTok{(numero_iteraciones,numero_parametros))}
  \KeywordTok{colnames}\NormalTok{(parametros)<-}\KeywordTok{c}\NormalTok{(}\KeywordTok{paste0}\NormalTok{(}\StringTok{"alpha"}\NormalTok{,}\DecValTok{1}\OperatorTok{:}\DecValTok{30}\NormalTok{),}\KeywordTok{paste0}\NormalTok{(}\StringTok{"beta"}\NormalTok{,}\DecValTok{1}\OperatorTok{:}\DecValTok{30}\NormalTok{),}\StringTok{"sigma_2_y"}\NormalTok{,}\StringTok{"sigma2_alpha"}\NormalTok{,}\StringTok{"sigma2_betas"}\NormalTok{,}\StringTok{"mu_alpha"}\NormalTok{,}\StringTok{"mu_beta"}\NormalTok{)}
\NormalTok{  parametros[}\DecValTok{1}\NormalTok{,] <-}\StringTok{ }\KeywordTok{c}\NormalTok{(alpha,beta,sigma2_y,sigma2_alpha,sigma2_beta,mu_alpha,mu_beta)}
  \ControlFlowTok{for}\NormalTok{(i }\ControlFlowTok{in} \DecValTok{1}\OperatorTok{:}\NormalTok{numero_iteraciones)\{}
    \ControlFlowTok{for}\NormalTok{(j }\ControlFlowTok{in} \DecValTok{1}\OperatorTok{:}\KeywordTok{length}\NormalTok{(alpha))\{}
\NormalTok{      mu_alpha <-}\StringTok{ }\KeywordTok{mu_alpha_condicionada}\NormalTok{(datos,alpha,sigma2_alpha)}
\NormalTok{      mu_beta <-}\StringTok{ }\KeywordTok{mu_beta_condicionada}\NormalTok{(datos,beta,sigma2_beta)}
\NormalTok{      sigma2_y <-}\StringTok{ }\KeywordTok{sigma2_y_condicionada}\NormalTok{(datos,alpha,beta,tiempos)}
\NormalTok{      sigma2_alpha <-}\StringTok{ }\KeywordTok{sigma2_alpha_condicionada}\NormalTok{(datos,alpha,mu_alpha)}
\NormalTok{      sigma2_beta <-}\StringTok{ }\KeywordTok{sigma2_beta_condicionada}\NormalTok{(datos,beta,mu_beta)}
\NormalTok{      alpha[j] <-}\StringTok{ }\KeywordTok{alpha_condicionada}\NormalTok{(mu_alpha,sigma2_alpha,datos[j,],beta[j],tiempos)}
\NormalTok{      beta[j] <-}\StringTok{ }\KeywordTok{beta_condicionada}\NormalTok{(mu_beta,sigma2_beta,datos[j,],alpha[j],tiempos)}
\NormalTok{    \}}
\NormalTok{    parametros_nuevos <-}\StringTok{ }\KeywordTok{c}\NormalTok{(alpha,beta,sigma2_y,sigma2_alpha,sigma2_beta,mu_alpha,mu_beta)}
\NormalTok{    parametros[i,] <-}\StringTok{ }\NormalTok{parametros_nuevos}
\NormalTok{  \}}
  \KeywordTok{return}\NormalTok{(parametros)}
\NormalTok{\}}
\end{Highlighting}
\end{Shaded}

\subsubsection{Resultados:}\label{resultados}

Obtenemos los siguientes resultados para 500 iteraciones y quemando el
\(10\%\) de las primeras iteraciones, en un tiempo de ejecución de
\textbf{5.2303181} segundos.

\begin{itemize}
\tightlist
\item
  El vector medio de \(\alpha\) (El intercept) sería:
\end{itemize}

\begin{verbatim}
##   alpha1   alpha2   alpha3   alpha4   alpha5   alpha6   alpha7   alpha8 
## 240.7105 248.9210 253.7090 233.1136 232.1195 250.9230 229.1376 249.5704 
##   alpha9  alpha10  alpha11  alpha12  alpha13  alpha14  alpha15  alpha16 
## 285.6516 219.3130 259.7480 228.5511 243.2812 270.0692 243.7078 246.3105 
##  alpha17  alpha18  alpha19  alpha20  alpha21  alpha22  alpha23  alpha24 
## 232.7511 241.2911 255.1427 242.5223 249.6881 225.4852 228.9416 246.1091 
##  alpha25  alpha26  alpha27  alpha28  alpha29  alpha30 
## 235.1137 255.3123 255.7055 243.9022 217.9804 242.3201
\end{verbatim}

\begin{itemize}
\tightlist
\item
  El vector medio de \(\beta\) , (La pendiente) sería:
\end{itemize}

\begin{verbatim}
##    beta1    beta2    beta3    beta4    beta5    beta6    beta7    beta8 
## 6.026748 7.307014 6.568257 5.092684 6.681338 6.171804 5.920192 6.482920 
##    beta9   beta10   beta11   beta12   beta13   beta14   beta15   beta16 
## 7.310165 5.746587 6.979658 6.105322 6.158245 6.841061 5.190020 5.842940 
##   beta17   beta18   beta19   beta20   beta21   beta22   beta23   beta24 
## 6.298896 5.746226 6.470086 6.014586 6.468520 5.764092 5.619896 5.803626 
##   beta25   beta26   beta27   beta28   beta29   beta30 
## 7.119033 6.653266 5.817402 5.742914 5.517246 6.112961
\end{verbatim}

\begin{itemize}
\tightlist
\item
  La desviación \(\sigma_y^2\) sería:
\end{itemize}

\begin{verbatim}
## [1] 23.24535
\end{verbatim}

\paragraph{Gráficas de convergencia de los parámetros en las
iteraciones:}\label{graficas-de-convergencia-de-los-parametros-en-las-iteraciones}

\begin{itemize}
\item
  \textbf{\(\mu_\alpha\)}:\\
  \includegraphics{Gibbs_algotithm_files/figure-latex/unnamed-chunk-16-1.pdf}
\item
  \textbf{\(\mu_\beta\)}:\\
  \includegraphics{Gibbs_algotithm_files/figure-latex/unnamed-chunk-17-1.pdf}
\end{itemize}

Podemos ver como rápidamente los parámetros convergen al resultado final
sin necesidad de muchas iteraciones.

\subsection{Inferencias}\label{inferencias}

\subsubsection{\texorpdfstring{Inferencias sobe el parámetro del término
independiente
(\(\alpha\)):}{Inferencias sobe el parámetro del término independiente (\textbackslash{}alpha):}}\label{inferencias-sobe-el-parametro-del-termino-independiente-alpha}

Bandas de credibilidad del 95 \% para alpha: Bandas de confianza para
:\\
\[\alpha_i \sim Normal(\mu_\alpha,\sigma_\alpha^2),\;\;\;i = 1..30\]
Cuantiles de \(\alpha_i\) 0.025 y 0.0975.\\
Podemos decir con una credibilidad del 95 \% que los valores de
\(\alpha_i\) para cada una de las 30 ratas están entre:

\begin{verbatim}
##            2.5%    97.5%
## rata1  239.6873 241.6172
## rata2  247.9963 249.9278
## rata3  252.7677 254.7505
## rata4  232.1745 234.1012
## rata5  231.2383 233.0864
## rata6  249.9678 251.9364
## rata7  228.0143 230.0301
## rata8  248.5658 250.6925
## rata9  284.7312 286.7944
## rata10 218.3562 220.4841
## rata11 258.7564 260.7256
## rata12 227.5919 229.6565
## rata13 242.3678 244.3364
## rata14 269.0845 271.0207
## rata15 242.7306 244.7709
## rata16 245.2914 247.3241
## rata17 231.7535 233.7130
## rata18 240.4152 242.2092
## rata19 254.2077 256.0408
## rata20 241.5127 243.5379
## rata21 248.7344 250.6350
## rata22 224.5199 226.4404
## rata23 228.0738 229.9363
## rata24 245.1106 247.0782
## rata25 234.0993 236.1159
## rata26 254.2744 256.1833
## rata27 254.6714 256.7125
## rata28 243.0196 244.8288
## rata29 216.9257 219.0321
## rata30 241.4361 243.2993
\end{verbatim}

Un gráfico que puede representar esto sería:\\
Donde se representa el Intervalo como una línea recta sólida horizontal
para cada una de las ratas.
\includegraphics{Gibbs_algotithm_files/figure-latex/unnamed-chunk-19-1.pdf}

\subsubsection{\texorpdfstring{Inferencias sobe el parámetro de la
pendiente
(\(\beta\)):}{Inferencias sobe el parámetro de la pendiente (\textbackslash{}beta):}}\label{inferencias-sobe-el-parametro-de-la-pendiente-beta}

Bandas de credibilidad del 95 \% para beta: Bandas de confianza para :\\
\[\beta_i \sim Normal(\mu_\beta,\sigma_\beta^2),\;\;\;i = 1..30\]
Cuantiles de \(\beta_i\) 0.025 y 0.0975.\\
Podemos decir con una credibilidad del 95 \% que los valores de
\(\beta_i\) para cada una de las 30 ratas están entre:

\begin{verbatim}
##            2.5%    97.5%
## rata1  5.941213 6.110255
## rata2  7.217817 7.393778
## rata3  6.487011 6.657705
## rata4  5.007271 5.187800
## rata5  6.582354 6.767558
## rata6  6.093528 6.247949
## rata7  5.832258 6.016020
## rata8  6.399591 6.561040
## rata9  7.221016 7.392690
## rata10 5.654192 5.835333
## rata11 6.895403 7.064241
## rata12 6.015129 6.189008
## rata13 6.071157 6.250649
## rata14 6.751769 6.922693
## rata15 5.102056 5.274068
## rata16 5.763167 5.937231
## rata17 6.206849 6.394546
## rata18 5.661884 5.829917
## rata19 6.382256 6.566349
## rata20 5.932988 6.100664
## rata21 6.380836 6.567719
## rata22 5.677527 5.845677
## rata23 5.535479 5.699236
## rata24 5.720397 5.894920
## rata25 7.038274 7.215053
## rata26 6.563770 6.738794
## rata27 5.728034 5.896947
## rata28 5.666596 5.835430
## rata29 5.430226 5.600996
## rata30 6.034870 6.190590
\end{verbatim}

Un gráfico que puede representar esto sería:\\
Donde se representa el Intervalo como una línea recta sólida horizontal
para cada una de las ratas.
\includegraphics{Gibbs_algotithm_files/figure-latex/unnamed-chunk-21-1.pdf}


\end{document}
